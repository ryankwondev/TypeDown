Very deep convolutional networks have been central to the largest advances in image
recognition performance in recent years. One example is the Inception
architecture that has been shown to achieve very good performance at relatively
low computational cost. Recently, the introduction of residual connections
in conjunction with a more traditional architecture has yielded state-of-the-art
performance in the 2015 ILSVRC challenge; its performance was similar
to the latest generation Inception-v3 network. This raises the question of whether
there are any benefit in combining the Inception architecture with residual
connections.
Here we give clear empirical evidence that training with residual connections
accelerates the training of Inception networks significantly. There is also
some evidence of residual Inception networks outperforming similarly
expensive Inception networks without residual connections by a thin margin.
We also present several new streamlined architectures for both residual and
non-residual Inception networks. These variations improve the single-frame
recognition performance on the ILSVRC 2012 classification task significantly.
We further demonstrate how proper activation scaling stabilizes the training of
very wide residual Inception networks. With an ensemble of three residual and
one Inception-v4, we achieve 3.08\% top-5 error on the test set of
the ImageNet classification (CLS) challenge.
